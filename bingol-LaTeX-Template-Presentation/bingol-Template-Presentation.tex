% !TEX spellcheck = en_US

\documentclass{beamer}
	\usetheme{Dresden}
%	\usecolortheme{}
%	\useinnertheme{}
%	\useoutertheme{}
%	\setbeamertemplate{bibliography item}[text]
\usepackage{beamerthemesplit}
%\usecolortheme{default}
%\usecolortheme{structure}
%\usepackage{beamerthemesplit} // Activate for custom appearance




%: ==== HB Header Common Packages v20160423 ====V
	\usepackage[utf8]{inputenc} % To use Unicode characters
	\usepackage[iso]{datetime}
	%	\usepackage{etex}
	
%	\usepackage[a4paper]{geometry}
	
	\usepackage{xcolor}
	% black, blue, brown, cyan, darkgray, gray, green, lightgray, lime, 
	% magenta, olive, orange, pink, purple, red, teal, violet, white, yellow
%	\usepackage[colorlinks=true,linkcolor=red,urlcolor=blue,citecolor=red]%
%		{hyperref}
	\usepackage{graphicx,epstopdf}
	% \graphicspath{{fig}}
%	\graphicspath{{../common/figures/}}
	% \DeclareGraphicsExtensions{.pdf,.jpeg,.png,.eps}
	% \DeclareGraphicsRule{.tif}{png}{.png}%
	%	{`convert #1 `dirname #1`/`basename #1 .tif`.png}
%	\usepackage{subfigure}
%\usepackage{subfig}
%\usepackage{subcaption}
%  ==== HB Header Common Packages v20160423 ==== A





% ======================================= V
%: ==== HB References
\newcommand{\hbDefined}[1]{\textcolor{red}{{#1}}}	
\newcommand{\hbEmph}[1]{\textcolor{orange}{#1}}
\newcommand{\hbReference}[1]{\textcolor{blue}{{\scriptsize [#1]}}}
\newcommand{\hbRemark}[1]{\textcolor{orange}{#1}}
\newcommand{\hbQuestion}[1]{\textcolor{orange}{Q. #1}}

% ==== separator
\newcommand{\hbSection}[2]{
	\section{{#1}} 
	\begin{frame}
		\hbRemark{{\center{\Huge {#1}}}}\\
		{#2}
	\end{frame}
}
\newcommand{\hbSeparator}[2]{
	\begin{frame}
		\hbRemark{{\center{\Huge {#1}}}}\\
		{#2}
	\end{frame}
}
% ======================================= A




% ==== HB page number ===================== V
\newcommand*\oldmacro{}%
\let\oldmacro\insertshorttitle%
\renewcommand*\insertshorttitle{%
 	\oldmacro\hfill%
	\insertframenumber\,/\,\inserttotalframenumber}
% ==== HB page number ===================== A




% ======================================= V
\title{
	Research in SoSLab
}

\author[Bingol]{Haluk O. Bingol}

\institute{
	Complex Systems research Lab (SoSLab)\\
	Dept. of Computer Engineering\\
	Bogazici University
}

\date{
	CMPE Conference\\ 
	Istanbul, 2019-04-08
}
% ======================================= A




\begin{document}
% ======================================= 




% ======================================= V Content
\frame{\titlepage}

%\frame{\tableofcontents}
\begin{frame}\frametitle{Content}
	\begin{columns}[T]
	\column{2in}
	\tableofcontents
	% You might wish to add the option [pausesections]
	\column{2.5in}
		\includegraphics[width=\columnwidth]%
%			{Fig-GossipLadies.png}\\
			{Fig-A.pdf}\\
	\end{columns}
\end{frame}
% ======================================= A




% =======================================
\hbSection{MyTitle}{MySubTitle}




% =======================================
\subsection{Single Column}




% ======================================= V
\begin{frame}\frametitle{Single Column}
	\begin{itemize}

		\item 
		``More is different''

		\item 
		Interacting many agents

		\item 
		Emergence

		\item 
		\hbReference{
			\href
				{https://www.cmpe.boun.edu.tr/courses/cmpe556}
				{cmpe556 Complex Systems}
		}
				
		\item
		\hbReference{
			\href
				{https://www.cmpe.boun.edu.tr/courses/cmpe557}
				{cmpe557 Complex Systems}
		}
				
		\item
		\hbReference{
			\href
				{https://www.cmpe.boun.edu.tr/courses/cmpe59e}
				{cmpe59e Evolutionary Dynamics}
		}
				
		\item
		\hbReference{
			\href
				{https://www.cmpe.boun.edu.tr/courses/cmpe59i}
				{cmpe59i Brain Dynamics}
		}
				
		\item
		\hbReference{
			\href
				{https://www.cmpe.boun.edu.tr/tr/courses/cmpe481}
				{cmpe481 Crowd Dynamics} (Networks, Crowds and Markets)
		}

	\end{itemize}
\end{frame}
% ======================================= A




% =======================================
\subsection{Two-Column}




% ======================================= V
\begin{frame}\frametitle{Two-Column}
	\begin{columns}[T]
	\begin{column}{.5\linewidth}   
	    	\begin{itemize}
		
			\item
			Language
		
			\item
			Small memory
		
			\item
			Complex Networks/\\Community Detection
		
			\item
			Brain and Epilepsy
		\end{itemize}
	\end{column}
	\begin{column}{.5\linewidth}
	    	\begin{itemize}
		
			\item
			Models
		
			\item
			e-Learning
			
		\end{itemize}
	\end{column}
	\end{columns}
\end{frame}
% ======================================= A




% =======================================
\subsection{Language}




% =======================================
\subsection{2-Column with Image}




% ======================================= V
\begin{frame}\frametitle{2-Column with Image}
	\begin{columns}[T]
	\begin{column}{.5\linewidth}   
	    	Effects of small memory
		\begin{itemize}
			\item
			There are too many to remember
		\end{itemize}

		\includegraphics[width=.8\columnwidth]%
			{Fig-B}\\
		\hbReference{
			\href
				{https://dilbert.com/}
				{Dilbert}
		}
			
	\end{column}
	\begin{column}{.5\linewidth}
	    	Questions
		\begin{itemize}
		
			\item
			Which one to learn
		
			\item
			Which one to forget
			
			\item
			What if memory gets smaller?
			
			\item
			Is big memory always good?
			
		\end{itemize}
	\end{column}
	\end{columns}
	
	\hbRemark{papers~\cite{%
		bingol2005LNCS,%
		bingol2008PRE,%
		delipinar2009SRM,%
		cetin2014PRE,%
		cetin2014CMP,%
		cetin2016ACS%
	}}
\end{frame}
% ======================================= A




% =======================================
\hbSection{Step by Step}




% ======================================= V
\begin{frame}\frametitle{Step by Step}

	The $1,000,000$ Question:  Backstage
	\begin{enumerate}[A]
		\item<2-5> 
		James Madison
		
		\item<3-5> 
		Harry Truman
		
		\item<4-> 
		\color<6>[rgb]{0,0.6,0}Abraham Lincoln
		
		\item<5-5>  
		Calvin Coolidge
	\end{enumerate}
	
	\uncover<1-5>{Hints:}\\
	\uncover<2-5>{James Madison ate broccoli.}\\
	\uncover<3-5>{Harry Truman drank milk.}\\
	\uncover<4-5>{Abe Lincoln raised bees.}\\
	\uncover<5-5>{And Cal Coolidge grew silk.}\\

\end{frame}
% ======================================= A

\section{aab}
\AtBeginSection[]{
	\begin{frame}{Table of Contents}
		\tableofcontents[currentsection]
		aab
	\end{frame}
}




% ======================================= V
\begin{frame}\frametitle{aaa}
	\begin{table}[bt]
	\begin{tabular}{|l|c|c|} 
		\hline
		\textbf{Ice Cream Store}      & \textbf{Location}& \textbf{How to Get There}   \\ 
		\hline
		\uncover<2->{Toscanini’s}	& \uncover<2->{Central Square}	& \uncover<2->{Just walk!}    \\
		\uncover<3->{Herrell’s}	& \uncover<3->{Harvard Square}	& \uncover<3->{Red Line}      \\
		\uncover<4->{J.P. Licks}	& \uncover<4->{Davis Square}	& \uncover<4->{Red Line}      \\
		\uncover<5->{Ben \& Jerry’s}	& \uncover<5->{Newbury Street}	& \uncover<5->{Green Line}    \\ 
		\hline
	\end{tabular}
	\end{table}
\end{frame}
% ======================================= A




% ======================================= References
\section*{References}




% ======================================= V References bib
\begin{frame}[allowframebreaks]\frametitle{References}
	\setbeamertemplate{bibliography item}[text] % sets numbers in references
%	\bibliographystyle{abbrv}
%	\bibliographystyle{apalike}
	\bibliographystyle{ieeetr}
	\tiny\bibliography{bingol-Template-Presentation}
	
\end{frame}
% ======================================= A





% ======================================= 
\end{document}
