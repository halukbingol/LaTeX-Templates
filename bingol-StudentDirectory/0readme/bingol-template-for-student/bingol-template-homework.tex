% !TEX spellcheck = en_US




%: ~~~~~~~~~~~~~~~~~~~~~~~~~~~~~~~~~~~~~~~ V license
% (NC) 2019 Haluk Bingol
% github.com/halukbingol/LaTeX-Templates
%
% Licensees may copy, distribute, display, and perform the work and make derivative works 
% and remixes based on it only for non-commercial purposes. 
% ~~~~~~~~~~~~~~~~~~~~~~~~~~~~~~~~~~~~~~~ A




% ~~~~~~~~~~~~~~~~~~~~~~~~~~~~~~~~~~~~~~~ 
\documentclass[11pt,a4]{article}




%: ~~~~~~~~~~~~~~~~~~~~~~~~~~~~~~~~~~~~~~~ V **** change every time
\newcommand{\hbWhat}{m1q3}
% ~~~~~~~~~~~~~~~~~~~~~~~~~~~~~~~~~~~~~~~ A




%: ~~~~~~~~~~~~~~~~~~~~~~~~~~~~~~~~~~~~~~~ V **** change once
\newcommand{\hbYou}{BingolHaluk}
\newcommand{\hbCourse}{cmpe220-2020-1}
% ~~~~~~~~~~~~~~~~~~~~~~~~~~~~~~~~~~~~~~~ A




%: ~~~~~~~~~~~~~~~~~~~~~~~~~~~~~~~~~~~~~~~ V package
\usepackage[utf8]{inputenc}
%
\usepackage{amssymb}
\usepackage{amsthm}
\usepackage[cmex10]{amsmath}
\usepackage{hyperref}
\usepackage[a4paper,top=2.3cm,bottom=4cm,left=2cm,right=4cm]{geometry}	% modified for electronic grading
\usepackage{graphicx}
\usepackage{xcolor}
\usepackage[caption=false,font=footnotesize]{subfig}
\usepackage{enumitem}
%
\usepackage[iso]{datetime}
\newcommand{\hbTimeStamp}{v\today/\currenttime} % version
%
\usepackage{lastpage}	% last page
\usepackage{fancyhdr}	% header/footer
\pagestyle{fancy}
%\lhead{aaa} 
%\chead{aaa}
\rhead{\hbCourse-\hbWhat-\hbYou}
\lfoot{homework}
\cfoot{\thepage/\pageref{LastPage}}
\rfoot{\hbTimeStamp} 
\renewcommand{\headrulewidth}{0.4pt} 
\renewcommand{\footrulewidth}{0.4pt}
%
\usepackage{lipsum}  % dummy text
% ~~~~~~~~~~~~~~~~~~~~~~~~~~~~~~~~~~~~~~~ A




%: ~~~~~~~~~~~~~~~~~~~~~~~~~~~~~~~~~~~~~~~ V title
\title{
	\hbCourse-\hbWhat-\hbYou\\
	{\small 	
		\hbTimeStamp\\
	}
}
\date{}
%
\begin{document}
\maketitle
% ~~~~~~~~~~~~~~~~~~~~~~~~~~~~~~~~~~~~~~~ A




%: ~~~~~~~~~~~~~~~~~~~~~~~~~~~~~~~~~~~~~~~ V README
{\color{red} % ~~~~
	\textbf{README.}
	This is how to use documentation for file submission and reporting.
	Read it.
	Then make sure that you remove this part.

	\emph{
		Note that the name of this file does not comply 
		the file naming convention described below. 
		Do not forget to change it.
	}
	
	\begin{itemize}
		
		\item 
		Use A-B-C-D-E-X.F the file naming convention for all of your files,
		e.g., ``cmpe220-2020-1-m1q1-BingolHaluk.tex'',
		where:
		\begin{description}
			
			\item[A]
			Course code, e.g. ``cmpe220'' 
			
			\item[B]
			Year in 4 digits, e.g. ``2020'' 
			
			\item[C] 
			Semester in code: ``1'': Spring, ``2'': Summer, ``3'': Fall.
			
			\item[D] 
			Assignment description, e.g., ``m1q3'', ``hw02'' or ``project''.
			
			\item[E]
			Your last name and first name in camel case, e.g., ``BingolHaluk.tex''.
			
			\item[X]
			If there is only one file, do not use this.
			If there are more than one file in your submission, 
			then you decide for a distinguisher.
			For example,\\
			``cmpe220-2020-1-m1q1-BingolHaluk-paper.tex'',\\
			``cmpe220-2020-1-m1q1-BingolHaluk-paper-fig1.pdf'',\\
			``cmpe220-2020-1-m1q1-BingolHaluk-paper-fig2.pdf'',\\
			``cmpe220-2020-1-m1q1-BingolHaluk-paper.bib'',\\
			``cmpe220-2020-1-m1q1-BingolHaluk-code-simulator.py'',\\
			``cmpe220-2020-1-m1q1-BingolHaluk-code-dataAnalysis.py''.
			
			\item[F]
			File extension, e.g., ``tex'', ``py''.
		\end{description}
		
		\item 
		Zip all the necessary files using the file convention,
		e.g., ``cmpe220-2020-1-m1q1-BingolHaluk.zip''.
		
		\item 
		Please submit your zip file to Moodle (moodle.boun.edu.tr) on time.
		
	\end{itemize}
} % color ~~~~
% ~~~~~~~~~~~~~~~~~~~~~~~~~~~~~~~~~~~~~~~ A




% ~~~~~~~~~~~~~~~~~~~~~~~~~~~~~~~~~~~~~~~ 
\section{aaa} 

\textbf{
Your text here.
Remove this.
}
\lipsum[2-3]




% ~~~~~~~~~~~~~~~~~~~~~~~~~~~~~~~~~~~~~~~ 
\subsection{bbb}

\textbf{
Your text here.
Remove this.
}
\lipsum[2-3]




% ~~~~~~~~~~~~~~~~~~~~~~~~~~~~~~~~~~~~~~~ 
\subsection{ccc}

\textbf{
Your text here.
Remove this.
}
\lipsum[2-3]




% ~~~~~~~~~~~~~~~~~~~~~~~~~~~~~~~~~~~~~~~ 
\end{document}