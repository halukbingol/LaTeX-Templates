% !TEX spellcheck = en_US




%: ~~~~~~~~~~~~~~~~~~~~~~~~~~~~~~~~~~~~~~~ V license
% (NC) 2019 Haluk Bingol
% github.com/halukbingol/LaTeX-Templates
%
% Licensees may copy, distribute, display, and perform the work and make derivative works 
% and remixes based on it only for non-commercial purposes. 
% ~~~~~~~~~~~~~~~~~~~~~~~~~~~~~~~~~~~~~~~ A




% ~~~~~~~~~~~~~~~~~~~~~~~~~~~~~~~~~~~~~~~ 
\documentclass[11pt,a4,twocolumn]{article}




% ~~~~~~~~~~~~~~~~~~~~~~~~~~~~~~~~~~~~~~ V specific 
% ~~~~~~~~~~~~~~~~~~~~~~~~~~~~~~~~~~~~~~ A




%: ~~~~~~~~~~~~~~~~~~~~~~~~~~~~~~~~~~~~~~~ V HB Packages v2021-04-01
	\usepackage[utf8]{inputenc} 
	\usepackage[iso]{datetime}
	\newcommand{\hbTimeStamp}{{\color{red}v\today T\currenttime}} % version
	\usepackage{enumerate}
	
	\usepackage[a4paper]{geometry}
	
	\usepackage{xcolor}
	% black, blue, brown, cyan, darkgray, gray, green, lightgray, lime, 
	% magenta, olive, orange, pink, purple, red, teal, violet, white, yellow
		\definecolor{darkred}{rgb}{0.8,0.1,0.1}
		\definecolor{darkgreen}{rgb}{0,0.5,0}
		\definecolor{darkblue}{rgb}{0,0,0.5}
		\colorlet{RED}{red}
	
	\usepackage[colorlinks=true,linkcolor=red,urlcolor=blue,citecolor=red]%
		{hyperref}
	\usepackage{graphicx,epstopdf}
	% \graphicspath{{fig}}
%	\graphicspath{{../common/figures/}}
	% \DeclareGraphicsExtensions{.pdf,.jpeg,.png,.eps}
	% \DeclareGraphicsRule{.tif}{png}{.png}%
	%	{`convert #1 `dirname #1`/`basename #1 .tif`.png}
%	\usepackage{subfigure}
%	\usepackage{subfig}
	\usepackage{subcaption}
% ~~~~~~~~~~~~~~~~~~~~~~~~~~~~~~~~~~~~~~~ A




%: ~~~~~~~~~~~~~~~~~~~~~~~~~~~~~~~~~~~~~~~ V HB Declarations v2021-04-01
	\newcommand{\reffig}[1]{Fig.~\ref{#1}}
	\newcommand{\refeq}[1]{Eq.~\ref{#1}}
	\newcommand{\reftbl}[1]{Table~\ref{#1}}
	\newcommand{\refsec}[1]{Sec.~\ref{#1}}
	\newcommand{\refcite}[1]{Ref~\cite{#1}}
	\newcommand{\refalg}[1]{Algorithm~\ref{#1}}
	\newcommand{\reflst}[1]{List.~\ref{#1}}  % code listing
	%
	\newcommand{\refthm}[1]{Theorem~\ref{#1}}
	\newcommand{\refthmA}[2]{\refthm{#1}(\ref{#2}}
	\newcommand{\reflem}[1]{Lemma~\ref{#1}}
	\newcommand{\refdef}[1]{Definition~\ref{#1}}
	\newcommand{\refexmp}[1]{Example~\ref{#1}}
	%
	\newcommand{\hQuote}[1]{{\small \textsf{``#1''}}}
	\newcommand{\hCode}[1]{\texttt{#1}}
	\newcommand{\hIdea}[1]{{\color{olive}{\scriptsize [{#1}]}}}
	\newcommand{\hFootnote}[2]{\footnote{{\color{red} @#1 : }#2}}
% ~~~~~~~~~~~~~~~~~~~~~~~~~~~~~~~~~~~~~~~ A




%: ~~~~~~~~~~~~~~~~~~~~~~~~~~~~~~~~~~~~~~~ V HB Math v2021-04-01
	\usepackage{amsmath, amssymb,amsfonts,amsthm}
	%
	\theoremstyle{plain}
	\newtheorem{thm}{Theorem}[section]
	\newtheorem{lem}[thm]{Lemma}
	\newtheorem{prop}[thm]{Proposition}
	\newtheorem*{cor}{Corollary}
	\theoremstyle{definition}
	\newtheorem{defn}{Definition}[section]
	\newtheorem{conj}{Conjecture}[section]
	\newtheorem{exmp}{Example}[section]
	\theoremstyle{remark}
	\newtheorem*{rem}{Remark}
	\newtheorem*{note}{Note}
	%
	\newcommand{\hDefined}[1]{\textcolor{darkred}{\textit{#1}}}	
	\newcommand{\hVec}[1]{\mathbf{#1}}	 
	\newcommand{\hAbs}[1]{\ensuremath{\left \lvert \, #1 \, \right \rvert} } % |x|
	\newcommand{\hMat}[1]{\mathbf{#1}}
	\newcommand{\hArgmin}[2]{\underset{#1}{\operatorname{arg \, min}}\;#2}
	\newcommand{\hArgmax}[2]{\underset{#1}{\operatorname{arg \, max}}\;#2}
% ~~~~~~~~~~~~~~~~~~~~~~~~~~~~~~~~~~~~~~~ A




% ~~~~~~~~~~~~~~~~~~~~~~~~~~~~~~~~~~~~~~~ V running header
\usepackage{lastpage}	% last page
%
\usepackage{fancyhdr}
\pagestyle{fancy}
%\lhead{aaa}
%\chead{bbb}
%\rhead{ccc}
\lfoot{Rules}
\cfoot{\thepage/\pageref{LastPage}}
\rfoot{\hbTimeStamp}
\renewcommand{\headrulewidth}{0.4pt} 
\renewcommand{\footrulewidth}{0.4pt}
% ~~~~~~~~~~~~~~~~~~~~~~~~~~~~~~~~~~~~~~~ A




% ~~~~~~~~~~~~~~~~~~~~~~~~~~~~~~~~~~~~~~~ V
\usepackage{lipsum}  % dummy text
%\lipsum[1-4]
% ~~~~~~~~~~~~~~~~~~~~~~~~~~~~~~~~~~~~~~~ A




% ~~~~~~~~~~~~~~~~~~~~~~~~~~~~~~~~~~~~~~~ V
\title{
	Rules
}
\author{Haluk O. Bingol}
\date{\hbTimeStamp}
%\date{\today}

\begin{document}
\maketitle
% ~~~~~~~~~~~~~~~~~~~~~~~~~~~~~~~~~~~~~~~ A





% ~~~~~~~~~~~~~~~~~~~~~~~~~~~~~~~~~~~~~~
%\appendix



% ~~~~~~~~~~~~~~~~~~~~~~~~~~~~~~~~~~~~~~
\section{General}

Use \LaTeX\ for reporting.




% ~~~~~~~~~~~~~~~~~~~~~~~~~~~~~~~~~~~~~~
\subsection{Naming convention for files}

Use the file naming convention in the form of ``$A$-$B$-$C$-$D$-$E$-$X$.$F$'' for all of your files,
e.g., ``cmpe220-2020-1-m1q1-BingolHaluk.tex'',
where:
\begin{description}
	
	\item[$A$]
	Course code, 
	e.g. ``cmpe220''.
	
	\item[$B$]
	Year in 4 digits, 
	e.g. ``2020'' .
	
	\item[$C$] 
	Semester in code: ``1'': Spring, ``2'': Summer, ``3'': Fall.
	
	\item[$D$] 
	Assignment description, 
	e.g., ``m1q3'', or ``hw02''.
	
	\item[$E$]
	Your last name and first name in camel case,
	e.g., ``BingolHaluk.tex''.
	
	\item[$X$]
	If there is only one file, do not use this.
	If there are more than one file in your submission, 
	then you decide for a distinguisher,
	e.g., ``fig1'', ``simulator''.
	See \reffig{fig:sampleContent}.
	
	\item[$F$]
	File extension, 
	e.g., ``tex'', ``py'', ``pdf''.
\end{description}





% +++++++++++++++++++++++++++++++++++++++V
%: -fig.1
\begin{figure}[!tbp]
	\centering 
	\hCode{cmpe220-2020-1-m1q1-BingolHaluk-paper.tex},\\
	\hCode{cmpe220-2020-1-m1q1-BingolHaluk-paper-fig1.pdf},\\
	\hCode{cmpe220-2020-1-m1q1-BingolHaluk-paper-fig2.pdf},\\
	\hCode{cmpe220-2020-1-m1q1-BingolHaluk-paper.bib},\\
	\hCode{cmpe220-2020-1-m1q1-BingolHaluk-code-simulator.py},\\
	\hCode{cmpe220-2020-1-m1q1-BingolHaluk-code-plots.py}.
	\caption{
		A sample content for project submission.
	} 
	% \caption
	\label{fig:sampleContent}
\end{figure}
% +++++++++++++++++++++++++++++++++++++++A




% ~~~~~~~~~~~~~~~~~~~~~~~~~~~~~~~~~~~~~~
\section{Moodle submission}

Your submission should be a single \hCode{.zip} file via Moodle.
All submissions are due 
specified date and time, sharp.




% ~~~~~~~~~~~~~~~~~~~~~~~~~~~~~~~~~~~~~~
\subsection{The \hCode{.zip} file}

Collect all necessary files in one \hCode{.zip} file.
The name should be ``$A$-$B$-$C$-$D$-$E$.zip''. 

Your zip file should contain the following:
\begin{enumerate}[i.]
	
	\item 
	\textbf{Report.}
	Your report in ``pdf'' format.
	
	\item
	\textbf{\LaTeX.}
	The necessary \LaTeX\ files to generate your report
	such as  
	\hCode{.tex}, 
	\hCode{.bib} and
	figure files.
	Do not include generated files such as \hCode{.log} or \hCode{.aux}.
		
	\item
	\textbf{Codes.}
	Your code files. 
	Use your favorite language.
	
	\item
	\textbf{Data.}
	Do not include data files.
	They maybe huge.
	
	\item
	Anything else that you think it is necessary.
	
\end{enumerate}




% ~~~~~~~~~~~~~~~~~~~~~~~~~~~~~~~~~~~~~~
\section{\LaTeX\ specific}

If they exist, 
make sure that 
you \emph{update} the following fields,
in your \hCode{.tex} file:
\begin{itemize}
	
	\item 
	``hbYourName''
	
	\item 
	``hbCourse''
	
	\item 
	``hbSemester''.
	
\end{itemize}




% ~~~~~~~~~~~~~~~~~~~~~~~~~~~~~~~~~~~~~~
\section{Specifics}




% ~~~~~~~~~~~~~~~~~~~~~~~~~~~~~~~~~~~~~~
\subsection{Project proposal}

	\textbf{Length.} 
	One page is preferred. 
	No more than two pages.




% ~~~~~~~~~~~~~~~~~~~~~~~~~~~~~~~~~~~~~~
\subsection{Project report}

	\textbf{Length.} 
	A typical report is around 6 pages. 
	Please no more than 12 pages.
	Please do not put lots of figure to make it appear to be long.
	Do not forget that not the quantity, but the quality matters.




%% ~~~~~~~~~~~~~~~~~~~~~~~~~~~~~~~~~~~~~~
%\section{Dummy}
%
%\lipsum[1-4]




% ~~~~~~~~~~~~~~~~~~~~~~~~~~~~~~~~~~~~~~
\section{Source}
The source of this file can be found at
\href
	{https://github.com/halukbingol/LaTeX-Templates/blob/master/bingol-template-for-student/bingol-readme.tex}
	{https://github.com/halukbingol/LaTeX-Templates/bingol-template-for-student/bingol-readme.tex}.





\end{document}  